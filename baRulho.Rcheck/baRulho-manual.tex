\nonstopmode{}
\documentclass[letterpaper]{book}
\usepackage[times,inconsolata,hyper]{Rd}
\usepackage{makeidx}
\usepackage[utf8,latin1]{inputenc}
% \usepackage{graphicx} % @USE GRAPHICX@
\makeindex{}
\begin{document}
\chapter*{}
\begin{center}
{\textbf{\huge Package `baRulho'}}
\par\bigskip{\large \today}
\end{center}
\begin{description}
\raggedright{}
\item[Type]\AsIs{Package}
\item[Title]\AsIs{Quantifying Habitat-Induced Acoustic Signal Degradation}
\item[Version]\AsIs{1.0.0}
\item[Date]\AsIs{2020-01-17}
\item[Author]\AsIs{Marcelo Araya-Salas [aut, cre]}
\item[Maintainer]\AsIs{Marcelo Araya-Salas }\email{marceloa27@gmail.com}\AsIs{}
\item[Description]\AsIs{Intended to facilitate acoustic analysis of (animal) sound transmission experiments, which typically aim to quantify changes in signal structure when transmitted in a given habitat by broadcasting and re-recording animal sounds at increasing distances. The package offers a workflow with functions to prepare the data set for analysis as well as to calculate and visualize several degradation metrics, including blur ratio, signal-to-noise ratio, excess attenuation and envelope correlation among others (Dabelsteen et al 1993 <doi:10.1121/1.406682>).}
\item[License]\AsIs{GPL (>= 2)}
\item[Imports]\AsIs{pbapply, utils, stats, seewave, tuneR}
\item[Depends]\AsIs{R (>= 3.2.1), warbleR (>= 1.1.20)}
\item[LazyData]\AsIs{TRUE}
\item[URL]\AsIs{}\url{https://github.com/maRce10/baRulho}\AsIs{}
\item[BugReports]\AsIs{}\url{https://github.com/maRce10/baRulho/issues}\AsIs{}
\item[NeedsCompilation]\AsIs{no}
\item[Suggests]\AsIs{ggplot2, knitr, kableExtra, viridis, png}
\item[VignetteBuilder]\AsIs{knitr}
\item[RoxygenNote]\AsIs{7.0.2}
\item[Repository]\AsIs{CRAN}
\item[Language]\AsIs{en-US}
\end{description}
\Rdcontents{\R{} topics documented:}
\inputencoding{utf8}
\HeaderA{atmospheric\_attenuation}{Measure atmospheric attenuation and absorption of sound}{atmospheric.Rul.attenuation}
%
\begin{Description}\relax
\code{atmospheric\_attenuation} measures atmospheric attenuation and atmospheric absorption.
\end{Description}
%
\begin{Usage}
\begin{verbatim}
atmospheric_attenuation(f, temp, RH, p = 101325, 
formula = 1, spi = NULL, dist = NULL)
\end{verbatim}
\end{Usage}
%
\begin{Arguments}
\begin{ldescription}
\item[\code{f}] numeric vector of length 1 with frequency (in Hertz).

\item[\code{temp}] numeric vector of length 1 with frequency (in Celsius).

\item[\code{RH}] numeric vector of length 1 with relative humidity

\item[\code{p}] numeric vector of length 1 with ambient pressure in Pa (standard: 101325, default).

\item[\code{formula}] 1 = Bazley 1976, 2 = Rossing 2007 (p. 116, see details).

\item[\code{spi}] numeric vector of length 1 with the initial sound pressure in Pa. Required for calculating atmospheric absorption. Default is \code{NULL}.

\item[\code{dist}] numeric vector of length 1 with distance (m) over which a sound propagates. Required for calculating atmospheric absorption. Default is \code{NULL}.
\end{ldescription}
\end{Arguments}
%
\begin{Details}\relax
Calculate the atmospheric attenuation based on temperature, relative humidity, pressure and sound frequency. The function can applied to formulae based on:
\begin{itemize}

\item{} \code{1}: default. As used by Bazley (1976), Sound absorption in air at frequencies up to 100 kHz. NPL acoustics report Ac 74. 
\item{} \code{2}: as used by Rossing (2007), Handbook of Acoustics, Springer.

\end{itemize}

If 'spi' and 'dist' are supplied the function also returns the atmospheric absorption (in dB).
\end{Details}
%
\begin{Value}
Returns atmospheric attenuation (in dB/m) of sound based on supplied parameters. If 'spi' and 'dist' are supplied the function also returns atmospheric absorption (in dB).
\end{Value}
%
\begin{Author}\relax
Marcelo Araya-Salas (\email{marceloa27@gmail.com})
\end{Author}
%
\begin{References}\relax

Araya-Salas, M. (2020). baRulho: baRulho: quantifying habitat-induced degradation of (animal) acoustic signals in R. R package version 1.0.0

\end{References}
%
\begin{Examples}
\begin{ExampleCode}
{
# load example data
data("playback_est")

#' # remove ambient selections
playback_est <- playback_est[playback_est$signal.type != "ambient", ]

# measure atmospheric attenuation formula 1
atmospheric_attenuation(f = 20000, temp = 20, RH = 90, p = 88000, formula = 1)
}

\end{ExampleCode}
\end{Examples}
\inputencoding{utf8}
\HeaderA{baRulho}{baRulho: quantifying habitat-induced acoustic signal degradation}{baRulho}
%
\begin{Description}\relax
`baRulho` is a package intended to quantify habitat-induced degradation of (animal) acoustic signals.
\end{Description}
%
\begin{Details}\relax
The main features of the package are:
\begin{itemize}

\item{} Loops to apply tasks through acoustic signals referenced in an extended selection table
\item{} The comparison of playback signals re-recorded at different distances 

\end{itemize}

Most functions allow the parallelization of tasks, which distributes the tasks among several processors to improve computational efficiency.

License: GPL (>= 2)
\end{Details}
%
\begin{Author}\relax
Marcelo Araya-Salas

Maintainer: Marcelo Araya-Salas (\email{marceloa27@gmail.com})
\end{Author}
\inputencoding{utf8}
\HeaderA{blur\_ratio}{Measure blur ratio in the time domain}{blur.Rul.ratio}
%
\begin{Description}\relax
\code{blur\_ratio} measures blur ratio in signals referenced in an extended selection table.
\end{Description}
%
\begin{Usage}
\begin{verbatim}
blur_ratio(X, parallel = 1, pb = TRUE, method = 1, ssmooth = 200, 
msmooth = NULL, output = "est", img = FALSE, res = 150, hop.size = 11.6, wl = NULL, 
ovlp = 70, pal = reverse.gray.colors.2, collevels = seq(-60, 0, 5), dest.path = NULL)
\end{verbatim}
\end{Usage}
%
\begin{Arguments}
\begin{ldescription}
\item[\code{X}] object of class 'extended\_selection\_table' created by the function \code{\LinkA{selection\_table}{selection.Rul.table}} from the warbleR package. The object must include the following additional columns: 'signal.type', 'bottom.freq' and 'top.freq'.

\item[\code{parallel}] Numeric vector of length 1. Controls whether parallel computing is applied by specifying the number of cores to be used. Default is 1 (i.e. no parallel computing).

\item[\code{pb}] Logical argument to control if progress bar is shown. Default is \code{TRUE}.

\item[\code{method}] Numeric vector of length 1 to indicate the 'experimental design' for measuring envelope correlation. Two methods are available:
\begin{itemize}

\item{} \code{1}: compare all signals with their counterpart that was recorded at the closest distance to source (e.g. compare a signal recorded at 5m, 10m and 15m with its counterpart recorded at 1m). This is the default method. 
\item{} \code{2}: compare all signals with their counterpart recorded at the distance immediately before (e.g. a signal recorded at 10m compared with the same signal recorded at 5m, then signal recorded at 15m compared with same signal recorded at 10m and so on).

\end{itemize}


\item[\code{ssmooth}] Numeric vector of length 1 determining the length of the sliding window (in amplitude samples) used for a sum smooth for amplitude envelope calculation (used internally by \code{\LinkA{env}{env}}). Default is 200.

\item[\code{msmooth}] Numeric vector of length 2 to smooth the amplitude envelope with a mean sliding window for amplitude envelope calculation. The first element is the window length (in number of amplitude values) and the second one the window overlap (used internally by \code{\LinkA{env}{env}}).

\item[\code{output}] Character vector of length 1 to determine if an extended selection table ('est', default) or a list ("list") containing the extended selection table (first object in the list) and all (smoothed) wave envelopes (second object in the list) is returned. The envelope data can be used for plotting.

\item[\code{img}] Logical argument to control if image files in 'jpeg' format containing the images being compared and the corresponding envelopes are produced. Default is no images ( \code{FALSE}).

\item[\code{res}] Numeric argument of length 1. Controls image resolution. Default is 150 (faster) although 300 - 400 is recommended for publication/presentation quality.

\item[\code{hop.size}] A numeric vector of length 1 specifying the time window duration (in ms). Default is 11.6 ms, which is equivalent to 512 wl for a 44.1 kHz sampling rate. Ignored if 'wl' is supplied.

\item[\code{wl}] A numeric vector of length 1 specifying the window length of the spectrogram, default 
is NULL. If supplied, 'hop.size' is ignored.

\item[\code{ovlp}] Numeric vector of length 1 specifying the percent overlap between two 
consecutive windows, as in \code{\LinkA{spectro}{spectro}}. Only used when plotting. Default is 70. Applied to both spectra and spectrograms on image files.

\item[\code{pal}] A color palette function to be used to assign colors in the 
plot, as in \code{\LinkA{spectro}{spectro}}. Default is reverse.gray.colors.2.

\item[\code{collevels}] Numeric vector indicating a set of levels which are used to partition the amplitude range of the spectrogram (in dB) as in \code{\LinkA{spectro}{spectro}}. Default is \code{seq(-60, 0, 5)}.

\item[\code{dest.path}] Character string containing the directory path where the image files will be saved. If NULL (default) then the folder containing the sound files will be used instead.
\end{ldescription}
\end{Arguments}
%
\begin{Details}\relax
Blur ratio measures the degradation of sound as a function of the change in signal energy in the time domain as described by Dabelsteen et al (1993). Low values indicate low degradation of signals. The function measures the blur ratio on signals in which a reference playback has been re-recorded at different distances. Blur ratio is measured as the mismatch between amplitude envelopes (expressed as probability density functions) of the reference signal and the re-recorded signal. The function compares each signal type to the corresponding reference signal within the supplied frequency range (e.g. bandpass) of the reference signal ('bottom.freq' and 'top.freq' columns in 'X'). The 'signal.type' column must be used to tell the function to only compare signals belonging to the same category (e.g. song-types). Two methods for setting the experimental design are provided. All wave objects in the extended selection table must have the same sampling rate so the length of envelopes is comparable.
\end{Details}
%
\begin{Value}
Data frame similar to input data, but also includes two new columns ('reference' and 'blur.ratio')
with the reference signal and blur ratio values. If \code{img = TRUE} it also returns 1 image file (in 'jpeg' format) for each comparison showing spectrograms of both signals and the overlaid amplitude envelopes (as probability mass functions (PMF)). Spectrograms are shown within the frequency range of the reference signal and also show vertical lines with the start and end of signals to allow users to visually check alignment. If \code{output = 'list'} the output would be a list including the data frame just described and a data frame with envelopes (amplitude values) for all signals.
\end{Value}
%
\begin{Author}\relax
Marcelo Araya-Salas (\email{marceloa27@gmail.com})
\end{Author}
%
\begin{References}\relax

Dabelsteen, T., Larsen, O. N., \& Pedersen, S. B. (1993). Habitat-induced degradation of sound signals: Quantifying the effects of communication sounds and bird location on blur ratio, excess attenuation, and signal-to-noise ratio in blackbird song. The Journal of the Acoustical Society of America, 93(4), 2206.

Araya-Salas, M. (2020). baRulho: baRulho: quantifying habitat-induced degradation of (animal) acoustic signals in R. R package version 1.0.0

\end{References}
%
\begin{SeeAlso}\relax
\code{\LinkA{envelope\_correlation}{envelope.Rul.correlation}}, \code{\LinkA{spectral\_blur\_ratio}{spectral.Rul.blur.Rul.ratio}}
\end{SeeAlso}
%
\begin{Examples}
\begin{ExampleCode}
{
# load example data
data("playback_est")

# remove ambient selections
playback_est <- playback_est[playback_est$signal.type != "ambient", ]

# using method 1
blur_ratio(X = playback_est)

# using method 2
blur_ratio(X = playback_est, method = 2)
}

\end{ExampleCode}
\end{Examples}
\inputencoding{utf8}
\HeaderA{envelope\_correlation}{Measure amplitude envelope correlation}{envelope.Rul.correlation}
%
\begin{Description}\relax
\code{envelope\_correlation} measures amplitude envelope correlation of signals referenced in an extended selection table.
\end{Description}
%
\begin{Usage}
\begin{verbatim}
envelope_correlation(X, parallel = 1, pb = TRUE, method = 1, cor.method = "pearson", 
ssmooth = NULL, msmooth = NULL, hop.size = 11.6, wl = NULL, ovlp = 70)
\end{verbatim}
\end{Usage}
%
\begin{Arguments}
\begin{ldescription}
\item[\code{X}] object of class 'extended\_selection\_table' created by the function \code{\LinkA{selection\_table}{selection.Rul.table}} from the warbleR package.

\item[\code{parallel}] Numeric vector of length 1. Controls whether parallel computing is applied by specifying the number of cores to be used. Default is 1 (i.e. no parallel computing).
If \code{NULL} (default) then the current working directory is used.

\item[\code{pb}] Logical argument to control if progress bar is shown. Default is \code{TRUE}.

\item[\code{method}] Numeric vector of length 1 to indicate the 'experimental design' to measure amplitude envelope correlation. Two methods are available:
\begin{itemize}

\item{} \code{1}: compare all signals with their counterpart that was recorded at the closest distance to source (e.g. compare a signal recorded at 5m, 10m and 15m with its counterpart recorded at 1m). This is the default method. 
\item{} \code{2}: compare all signals with their counterpart recorded at the distance immediately before (e.g. a signal recorded at 10m compared with the same signal recorded at 5m, then signal recorded at 15m compared with same signal recorded at 10m and so on).

\end{itemize}


\item[\code{cor.method}] Character string indicating the correlation coefficient to be applied ("pearson", "spearman", or "kendall", see \code{\LinkA{cor}{cor}}).

\item[\code{ssmooth}] Numeric vector of length 1 to determine the length of the sliding window used for a sum smooth for amplitude envelope calculation (used internally by \code{\LinkA{env}{env}}).

\item[\code{msmooth}] Numeric vector of length 2 to smooth the amplitude envelope with a mean sliding window for amplitude envelope calculation. The first element is the window length (in number of amplitude values) and the second one the window overlap (used internally by \code{\LinkA{env}{env}}).

\item[\code{hop.size}] A numeric vector of length 1 specifying the time window duration (in ms). Default is 11.6 ms, which is equivalent to 512 wl for a 44.1 kHz sampling rate. Ignored if 'wl' is supplied.

\item[\code{wl}] A numeric vector of length 1 specifying the window length of the spectrogram, default 
is NULL. If supplied, 'hop.size' is ignored.

\item[\code{ovlp}] Numeric vector of length 1 specifying the percent overlap between two 
consecutive windows, as in \code{\LinkA{spectro}{spectro}}. Default is 70.
\end{ldescription}
\end{Arguments}
%
\begin{Details}\relax
Amplitude envelope correlation measures the similarity of two signals in the time domain. The  function measures the envelope correlation coefficients of signals in which a reference playback has been re-recorded at increasing distances. Values close to 1 means very similar amplitude envelopes (i.e. little degradation has occurred). If envelopes have different lengths (which means signals have different lengths) cross-correlation is used and the maximum correlation coefficient is returned. Cross-correlation is achieved by sliding the shortest signal along the largest one and calculating the correlation at each step. The 'signal.type' column must be used to indicate the function to only compare signals belonging to the same category (e.g. song-types).The function compares each signal type to the corresponding reference signal within the supplied frequency range (e.g. bandpass) of the reference signal ('bottom.freq' and 'top.freq' columns in 'X'). Two methods for calculating envelope correlation are provided (see 'method' argument). Use \code{\LinkA{blur\_ratio}{blur.Rul.ratio}} to extract envelopes.
\end{Details}
%
\begin{Value}
Extended selection table similar to input data, but also includes two new columns ('reference' and  'envelope.correlation')
with the reference signal and the amplitude envelope correlation coefficients.
\end{Value}
%
\begin{Author}\relax
Marcelo Araya-Salas (\email{marceloa27@gmail.com})
\end{Author}
%
\begin{References}\relax

Araya-Salas, M. (2020). baRulho: baRulho: quantifying habitat-induced degradation of (animal) acoustic signals in R. R package version 1.0.0

Apol, C.A., Sturdy, C.B. \& Proppe, D.S. (2017). Seasonal variability in habitat structure may have shaped acoustic signals and repertoires in the black-capped and boreal chickadees. Evol Ecol. 32:57-74.

\end{References}
%
\begin{SeeAlso}\relax
\code{\LinkA{blur\_ratio}{blur.Rul.ratio}}, \code{\LinkA{spectral\_blur\_ratio}{spectral.Rul.blur.Rul.ratio}}
\end{SeeAlso}
%
\begin{Examples}
\begin{ExampleCode}
{
# load example data
data("playback_est")

# remove ambient selections
playback_est <- playback_est[playback_est$signal.type != "ambient", ]

# method 1
envelope_correlation(X = playback_est)

# method 2
envelope_correlation(X = playback_est, method = 2)
}

\end{ExampleCode}
\end{Examples}
\inputencoding{utf8}
\HeaderA{excess\_attenuation}{Measure excess attenuation}{excess.Rul.attenuation}
%
\begin{Description}\relax
\code{excess\_attenuation} measures excess attenuation in signals referenced in an extended selection table.
\end{Description}
%
\begin{Usage}
\begin{verbatim}
excess_attenuation(X, parallel = 1, pb = TRUE, method = 1, 
bp = NULL, hop.size = 1, wl = NULL, ovlp = 70)
\end{verbatim}
\end{Usage}
%
\begin{Arguments}
\begin{ldescription}
\item[\code{X}] object of class 'extended\_selection\_table' created by the function \code{\LinkA{selection\_table}{selection.Rul.table}} from the warbleR package. The data frame must include the following additional columns: 'distance', 'signal.type', 'bottom.freq' and 'top.freq'.

\item[\code{parallel}] Numeric vector of length 1. Controls whether parallel computing is applied by specifying the number of cores to be used. Default is 1 (i.e. no parallel computing).

\item[\code{pb}] Logical argument to control if progress bar is shown. Default is \code{TRUE}.

\item[\code{method}] Numeric vector of length 1 to indicate the 'experimental design' for measuring excess attenuation. Two methods are available:
\begin{itemize}

\item{} \code{1}: compare all signals with their counterpart that was recorded at the closest distance to source (e.g. compare a signal recorded at 5m, 10m and 15m with its counterpart recorded at 1m). This is the default method. 
\item{} \code{2}: compare all signals with their counterpart recorded at the distance immediately before (e.g. a signal recorded at 10m compared with the same signal recorded at 5m, then signal recorded at 15m compared with same signal recorded at 10m and so on).

\end{itemize}


\item[\code{bp}] Numeric vector of length 2 giving the lower and upper limits of a frequency bandpass filter (in kHz). Default is \code{NULL}.

\item[\code{hop.size}] A numeric vector of length 1 specifying the time window duration (in ms). Default is 1 ms, which is equivalent to \textasciitilde{}45 wl for a 44.1 kHz sampling rate. Ignored if 'wl' is supplied.

\item[\code{wl}] A numeric vector of length 1 specifying the window length of the spectrogram, default 
is NULL. Ignored if \code{bp = NULL}. If supplied, 'hop.size' is ignored.
Note that lower values will increase time resolution, which is more important for amplitude ratio calculations.

\item[\code{ovlp}] Numeric vector of length 1 specifying the percent overlap between two 
consecutive windows, as in \code{\LinkA{spectro}{spectro}}. Only used when plotting. Default is 70. Only used for bandpass filtering.
\end{ldescription}
\end{Arguments}
%
\begin{Details}\relax
Excess attenuation is the amplitude loss of a sound in excess due to spherical spreading. With every doubling of distance, sounds attenuate with a 6 dB loss of amplitude (Morton, 1975; Marten \& Marler, 1977). Any additional loss of amplitude results in excess attenuation, or energy loss in excess of that expected to occur with distance via spherical spreading, due to atmospheric conditions or habitat (Wiley \& Richards, 1978). Low values indicate little signal attenuation. 
The goal of the function is to measure the excess attenuation on signals in which a reference playback has been re-recorded at increasing distances. The 'signal.type' column must be used to indicate which signals belonging to the same category (e.g. song-types). The function will then compare each signal type to the corresponding reference signal the supplied frequency range (e.g. bandpass) of the reference signal ('bottom.freq' and 'top.freq' columns in 'X'). Two methods for calculating excess attenuation are provided (see 'method' argument).
\end{Details}
%
\begin{Value}
Extended selection table similar to input data, but also includes a new column (excess.attenuation)
with the excess attenuation values.
\end{Value}
%
\begin{Author}\relax
Marcelo Araya-Salas (\email{marceloa27@gmail.com})
\end{Author}
%
\begin{References}\relax

Araya-Salas, M. (2020),=. baRulho: baRulho: quantifying habitat-induced degradation of (animal) acoustic signals in R. R package version 1.0.0

Marten, K., \& Marler, P. (1977). Sound transmission and its significance for animal vocalization. Behavioral Ecology and Sociobiology, 2(3), 271-290.

Morton, E. S. (1975). Ecological sources of selection on avian sounds. The American Naturalist, 109(965), 17-34.

\end{References}
%
\begin{SeeAlso}\relax
\code{\LinkA{spcc}{spcc}}
\end{SeeAlso}
%
\begin{Examples}
\begin{ExampleCode}
{
# load example data
data("playback_est")

# using method 1
excess_attenuation(X = playback_est)

# using method 2
excess_attenuation(X = playback_est, method = 2)
}

\end{ExampleCode}
\end{Examples}
\inputencoding{utf8}
\HeaderA{master\_sound\_file}{Create a master sound file}{master.Rul.sound.Rul.file}
%
\begin{Description}\relax
\code{master\_sound\_file} creates a master sound file to be used in playback experiments related to sound degradation.
\end{Description}
%
\begin{Usage}
\begin{verbatim}
master_sound_file(X, file.name, dest.path = NULL, overwrite = FALSE, delay = 1, 
gap.duration = 1, amp.marker = 2)
\end{verbatim}
\end{Usage}
%
\begin{Arguments}
\begin{ldescription}
\item[\code{X}] object of class 'extended\_selection\_table' created by the function \code{\LinkA{selection\_table}{selection.Rul.table}} from the warbleR package. The object must include the following additional columns: 'bottom.freq' and 'top.freq'.

\item[\code{file.name}] Character string indicating the name of the sound file.

\item[\code{dest.path}] Character string containing the directory path where the sound file will be saved.
If \code{NULL} (default) then the current working directory will be used instead.

\item[\code{overwrite}] Logical argument to determine if the function will overwrite any existing sound file with the same file name. Default is \code{FALSE}.

\item[\code{delay}] Numeric vector of length 1 to control the duration (in s) of a silence gap at the beginning of the sound file. This can be useful to allow some time at the start of the playback experiment. Default is 1.

\item[\code{gap.duration}] Numeric vector of length 1 to control the duration (in s) of silence gaps to be placed in between signals. Default is 1 s.

\item[\code{amp.marker}] Numeric vector of length 1 to use as a constant to amplify markers amplitude. This is useful to increase the amplitude of markers in relation to those of signals, so it is picked up at further distances. Default is 2.
\end{ldescription}
\end{Arguments}
%
\begin{Details}\relax
The function is intended to simplify the creation of master sound files for playback experiments in signal degradation studies. The function takes the wave objects from extended selection tables and concatenate them in a single sound file. The function also adds acoustic markers at the start and end of the playback that can be used to time-sync re-recorded signals to facilitate the streamlining of degradation quantification.
\end{Details}
%
\begin{Value}
Extended selection table similar to input data, but includes a new column (cross.correlation)
with the spectrogram cross-correlation coefficients.
\end{Value}
%
\begin{Author}\relax
Marcelo Araya-Salas (\email{marceloa27@gmail.com})
\end{Author}
%
\begin{References}\relax

Araya-Salas, M. (2020). baRulho: baRulho: quantifying habitat-induced degradation of (animal) acoustic signals in R. R package version 1.0.0

\end{References}
%
\begin{SeeAlso}\relax
\code{\LinkA{exp\_raven}{exp.Rul.raven}}
\end{SeeAlso}
%
\begin{Examples}
\begin{ExampleCode}
{
# load example data from warbleR
data(list = c("Phae.long1", "Phae.long2", "Phae.long3", "Phae.long4", 
"lbh_selec_table"))

# save sound files to temporary folder
writeWave(Phae.long1, file.path(tempdir(), "Phae.long1.wav"))
writeWave(Phae.long2, file.path(tempdir(), "Phae.long2.wav"))
writeWave(Phae.long3, file.path(tempdir(), "Phae.long3.wav"))
writeWave(Phae.long4, file.path(tempdir(), "Phae.long4.wav"))

# make an extended selection table
est <- selection_table(X = lbh_selec_table, extended = TRUE, confirm.extended = FALSE, 
path = tempdir())

# create master sound file
master.sel.tab <- master_sound_file(X = est, file.name = "example_master", 
dest.path = tempdir(), gap.duration = 0.3)

# the following code exports the selection table to Raven using Rraven package
# Rraven::exp_raven(master.sel.tab, path = tempdir(), file.name = "example_master_selection_table")
}

\end{ExampleCode}
\end{Examples}
\inputencoding{utf8}
\HeaderA{playback\_est}{Extended selection table with re-recorded playbacks}{playback.Rul.est}
\keyword{datasets}{playback\_est}
%
\begin{Description}\relax
Recordings of \emph{Phaethornis longirostris} (Long-billed Hermit) songs from different song types (column 'signal.type') that were broadcast and re-recorded at 4 distances (1m, 5m, 10m, 15m, column 'distance'). The data includes ambient (background) noise selections for each distances. The data was created by the function \code{\LinkA{selection\_table}{selection.Rul.table}} from the warbleR package.

Recordings of \emph{Phaethornis longirostris} (Long-billed Hermit) songs from different song types (column 'signal.id') that were broadcast and re-recorded at 4 distances (1m, 5m, 10m, 15m, column 'distance'). The data includes ambient (background) noise selections for each distances. The data was created by the function \code{\LinkA{selection\_table}{selection.Rul.table}} from the warbleR package.
\end{Description}
%
\begin{Usage}
\begin{verbatim}
data(playback_est)

data(playback_est)
\end{verbatim}
\end{Usage}
%
\begin{Format}
Extended selection table object in the \code{\LinkA{warbleR}{warbleR}} format, which contains annotations and acoustic data
\end{Format}
%
\begin{Source}\relax
Marcelo Araya-Salas

Marcelo Araya-Salas
\end{Source}
\inputencoding{utf8}
\HeaderA{playback\_est\_unaligned}{Extended selection table with re-recorded playbacks before alignment}{playback.Rul.est.Rul.unaligned}
\keyword{datasets}{playback\_est\_unaligned}
%
\begin{Description}\relax
The data contains a subet of the selections in the example data 'playback\_est' but in this subset the re-recorded signals are not aligned in time with the corresponding reference signals (see \code{\LinkA{spcc\_align}{spcc.Rul.align}} for more details on aligning signals). This data set is intended mostly for using as an example in \code{\LinkA{spcc\_align}{spcc.Rul.align}}. The data contains recordings of \emph{Phaethornis longirostris} (Long-billed Hermit) songs from different song types (column 'signal.type') that were broadcast and re-recorded at 4 distances (1m, 5m, 10m, 15m, column 'distance'). The data was created by the function \code{\LinkA{selection\_table}{selection.Rul.table}} from the \code{\LinkA{warbleR}{warbleR}} package.
\end{Description}
%
\begin{Usage}
\begin{verbatim}
data(playback_est_unaligned)
\end{verbatim}
\end{Usage}
%
\begin{Format}
Extended selection table object in the \code{\LinkA{warbleR}{warbleR}} format, which contains annotations and acoustic data
\end{Format}
%
\begin{Source}\relax
Marcelo Araya-Salas
\end{Source}
\inputencoding{utf8}
\HeaderA{snr}{Measure attenuation as signal-to-noise ratio}{snr}
%
\begin{Description}\relax
\code{snr} measures attenuation as signal-to-noise ratio of signals referenced in an extended selection table.
\end{Description}
%
\begin{Usage}
\begin{verbatim}
snr(X, mar, parallel = 1, pb = TRUE, eq.dur = FALSE,
noise.ref = "adjacent", type = 1, bp = NULL, hop.size = 1, wl = NULL)
\end{verbatim}
\end{Usage}
%
\begin{Arguments}
\begin{ldescription}
\item[\code{X}] object of class 'extended\_selection\_table' created by the function \code{\LinkA{selection\_table}{selection.Rul.table}} from the warbleR package.

\item[\code{mar}] numeric vector of length 1. Specifies the margins adjacent to
the start and end points of selection over which to measure ambient noise.

\item[\code{parallel}] Numeric vector of length 1. Controls whether parallel computing is applied by specifying the number of cores to be used. Default is 1 (i.e. no parallel computing).

\item[\code{pb}] Logical argument to control if progress bar is shown. Default is \code{TRUE}.

\item[\code{eq.dur}] Logical. Controls whether the ambient noise segment that is measured has the same duration 
to that of the signal (if \code{TRUE}. Default is \code{FALSE}). If \code{TRUE} then 'mar' and 'noise.ref' arguments are ignored.

\item[\code{noise.ref}] Character vector of length 1 to determined if a measure ambient noise segment to be used for measuring ambient noise. Two options are available: 
\begin{itemize}

\item{} \code{adjacent}: measure ambient noise right before the signal (using argument 'mar' to define duration of ambient noise segments). If several 'ambient' selections by sound file are supplied, then the root mean square of the amplitude envelope will be averaged across those selections.
\item{} \code{custom}: measure ambient noise segments referenced in the selection table (labeled as 'ambient' in the 'signal.type' column). Those segments will be used to apply the same ambient noise reference to all signals in a sound file. Therefore, at least one 'ambient' selection for each sound file must be provided.

\end{itemize}


\item[\code{type}] Numeric vector of length 1. Selects the formula to be used to calculate the signal-to-noise ratio (S = signal
, N = background noise): 
\begin{itemize}

\item{} \code{1}: ratio of S amplitude envelope root mean square to N amplitude envelope root mean square
(\code{rms(env(S))/rms(env(N))})
\item{} \code{2}: ratio of the difference between S amplitude envelope root mean square and N amplitude envelope root mean square to N amplitude envelope root mean square (\code{(rms(env(S)) - rms(env(N)))/rms(env(N))}, as proposed by Dabelsteen et al (1993))

\end{itemize}


\item[\code{bp}] Numeric vector of length 2 giving the lower and upper limits of a frequency bandpass filter (in kHz). Default is \code{NULL}.

\item[\code{hop.size}] A numeric vector of length 1 specifying the time window duration (in ms). Default is 1 ms, which is equivalent to \textasciitilde{}45 wl for a 44.1 kHz sampling rate. Ignored if 'wl' is supplied.

\item[\code{wl}] A numeric vector of length 1 specifying the window length of the spectrogram, default 
is NULL. Ignored if \code{bp = NULL}. If supplied, 'hop.size' is ignored.
Note that lower values will increase time resolution, which is more important for amplitude ratio calculations.
\end{ldescription}
\end{Arguments}
%
\begin{Details}\relax
Signal-to-noise ratio (SNR) measures signal amplitude level in relation to ambient noise. A general margin in which ambient noise will be measured must be specified. Alternatively, a selection of ambient noise can be used as reference (see 'noise.ref' argument). When margins overlap with another acoustic signal nearby, SNR will be inaccurate, so margin length should be carefully considered. Any SNR less than or equal to one suggests background noise is equal to or overpowering the acoustic signal. The 'signal.type' column must be used to indicate which signals belong to the same category (e.g. song-types). The function will measure signal-to-noise ratio within the supplied frequency range (e.g. bandpass) of the reference signal ('bottom.freq' and 'top.freq' columns in 'X'). Two methods for calculating signal-to-noise ratio are provided (see 'type' argument).
\end{Details}
%
\begin{Value}
Extended selection table similar to input data, but also includes a new column (snr.attenuation)
with the signal-to-noise ratio values.
\end{Value}
%
\begin{Author}\relax
Marcelo Araya-Salas (\email{marceloa27@gmail.com})
\end{Author}
%
\begin{References}\relax

Dabelsteen, T., Larsen, O. N., \& Pedersen, S. B. (1993). Habitat-induced degradation of sound signals: Quantifying the effects of communication sounds and bird location on blur ratio, excess attenuation, and signal-to-noise ratio in blackbird song. The Journal of the Acoustical Society of America, 93(4), 2206.

Araya-Salas, M. (2020). baRulho: baRulho: quantifying habitat-induced degradation of (animal) acoustic signals in R. R package version 1.0.0

\end{References}
%
\begin{SeeAlso}\relax
\code{\LinkA{excess\_attenuation}{excess.Rul.attenuation}}
\end{SeeAlso}
%
\begin{Examples}
\begin{ExampleCode}
{
# load example data
data("playback_est")

# using measure ambient noise reference selections 
snr(X = playback_est, mar = 0.05, noise.ref = 'custom')

# remove ambient selections
playback_est <- playback_est[playback_est$signal.type != "ambient", ]
# using margin for ambient noise of 0.05 and adjacent measure ambient noise reference
snr(X = playback_est, mar = 0.05, noise.ref = 'adjacent')
}

\end{ExampleCode}
\end{Examples}
\inputencoding{utf8}
\HeaderA{spcc}{Measure spectrographic cross-correlation as a measure of signal distortion}{spcc}
%
\begin{Description}\relax
\code{spcc} measures spectrographic cross-correlation as a measure of signal distortion in signals referenced in an extended selection table.
\end{Description}
%
\begin{Usage}
\begin{verbatim}
spcc(X, parallel = 1, pb = TRUE,  method = 1, 
cor.method = "pearson", hop.size = 11.6, wl = NULL, ovlp = 90, wn = 'hanning')
\end{verbatim}
\end{Usage}
%
\begin{Arguments}
\begin{ldescription}
\item[\code{X}] object of class 'extended\_selection\_table' created by the function \code{\LinkA{selection\_table}{selection.Rul.table}} from the warbleR package. The object must include the following additional columns: 'signal.type', 'bottom.freq' and 'top.freq'.

\item[\code{parallel}] Numeric vector of length 1. Controls whether parallel computing is applied by specifying the number of cores to be used. Default is 1 (i.e. no parallel computing).

\item[\code{pb}] Logical argument to control if progress bar is shown. Default is \code{TRUE}.

\item[\code{method}] Numeric vector of length 1 to indicate the 'experimental design' for measuring envelope correlation. Two methods are available:
\begin{itemize}

\item{} \code{1}: compare all signals with their counterpart that was recorded at the closest distance to source (e.g. compare a signal recorded at 5m, 10m and 15m with its counterpart recorded at 1m). This is the default method. 
\item{} \code{2}: compare all signals with their counterpart recorded at the distance immediately before (e.g. a signal recorded at 10m compared with the same signal recorded at 5m, then signal recorded at 15m compared with same signal recorded at 10m and so on).

\end{itemize}


\item[\code{cor.method}] Character string indicating the correlation coefficient to be applied ("pearson", "spearman", or "kendall", see \code{\LinkA{cor}{cor}}).

\item[\code{hop.size}] A numeric vector of length 1 specifying the time window duration (in ms). Default is 11.6 ms, which is equivalent to 512 wl for a 44.1 kHz sampling rate. Ignored if 'wl' is supplied.

\item[\code{wl}] A numeric vector of length 1 specifying the window length of the spectrogram, default 
is NULL. If supplied, 'hop.size' is ignored.

\item[\code{ovlp}] Numeric vector of length 1 specifying \% of overlap between two 
consecutive windows, as in \code{\LinkA{spectro}{spectro}}. Default is 90. High values of ovlp 
slow down the function but produce more accurate results.

\item[\code{wn}] A character vector of length 1 specifying the window name as in \code{\LinkA{ftwindow}{ftwindow}}.
\end{ldescription}
\end{Arguments}
%
\begin{Details}\relax
Spectrographic cross-correlation measures frequency distortion of signals as a similarity metric. Values close to 1 means very similar spectrograms (i.e. little signal distortion has occurred). Cross-correlation is measured of signals in which a reference playback has been re-recorded at increasing distances. The 'signal.type' column must be used to indicate the function to only compare signals belonging to the same category (e.g. song-types). The function compares each signal type to the corresponding reference signal within the supplied frequency range (e.g. bandpass) of the reference signal ('bottom.freq' and 'top.freq' columns in 'X'). Two methods for calculating cross-correlation are provided (see 'method' argument). The function is a wrapper on warbleR's \code{\LinkA{xcorr}{xcorr}} function.
\end{Details}
%
\begin{Value}
Extended selection table similar to input data, but includes a new column (cross.correlation)
with the spectrogram cross-correlation coefficients.
\end{Value}
%
\begin{Author}\relax
Marcelo Araya-Salas (\email{marceloa27@gmail.com})
\end{Author}
%
\begin{References}\relax

Araya-Salas, M. (2020). baRulho: baRulho: quantifying habitat-induced degradation of (animal) acoustic signals in R. R package version 1.0.0

Clark, C.W., Marler, P. \& Beeman K. (1987). Quantitative analysis of animal vocal phonology: an application to Swamp Sparrow song. Ethology. 76:101-115. 

\end{References}
%
\begin{SeeAlso}\relax
\code{\LinkA{blur\_ratio}{blur.Rul.ratio}}, \code{\LinkA{spcc\_align}{spcc.Rul.align}}, \code{\LinkA{xcorr}{xcorr}}
\end{SeeAlso}
%
\begin{Examples}
\begin{ExampleCode}
{
# load example data
data("playback_est")

# method 1
spcc(X = playback_est, method = 1)

# method 2
spcc(X = playback_est, method = 2)
}

\end{ExampleCode}
\end{Examples}
\inputencoding{utf8}
\HeaderA{spcc\_align}{Align start and end of signal using spectrographic cross-correlation}{spcc.Rul.align}
%
\begin{Description}\relax
\code{spcc\_align} aligns start and end of signal in an extended selection table using spectrographic cross-correlation
\end{Description}
%
\begin{Usage}
\begin{verbatim}
spcc_align(X, parallel = 1, pb = TRUE, hop.size = 11.6, wl = NULL, ovlp = 90, 
wn = 'hanning')
\end{verbatim}
\end{Usage}
%
\begin{Arguments}
\begin{ldescription}
\item[\code{X}] object of class 'extended\_selection\_table' created by the function \code{\LinkA{selection\_table}{selection.Rul.table}} from the warbleR package. The object must include the following additional columns: 'signal.type', 'bottom.freq' and 'top.freq'.

\item[\code{parallel}] Numeric vector of length 1. Controls whether parallel computing is applied by specifying the number of cores to be used. Default is 1 (i.e. no parallel computing).

\item[\code{pb}] Logical argument to control if progress bar is shown. Default is \code{TRUE}.

\item[\code{hop.size}] A numeric vector of length 1 specifying the time window duration (in ms). Default is 11.6 ms, which is equivalent to 512 wl for a 44.1 kHz sampling rate. Ignored if 'wl' is supplied.

\item[\code{wl}] A numeric vector of length 1 specifying the window length of the spectrogram, default 
is NULL. If supplied, 'hop.size' is ignored.

\item[\code{ovlp}] Numeric vector of length 1 specifying \% of overlap between two 
consecutive windows, as in \code{\LinkA{spectro}{spectro}}. Default is 90. High values of ovlp 
slow down the function but produce more accurate results.

\item[\code{wn}] A character vector of length 1 specifying the window name as in \code{\LinkA{ftwindow}{ftwindow}}.
\end{ldescription}
\end{Arguments}
%
\begin{Details}\relax
This function uses spectrographic cross-correlation to align the position in time of signals with regard to a reference signal. The signal recorded at the closest distance to the source is used as reference. Precise alignment is crucial for downstream measures of signal degradation. The function calls warbleR's \code{\LinkA{xcorr}{xcorr}} and \code{\LinkA{find\_peaks}{find.Rul.peaks}} internally to align signals using cross-correlation. The output extended selection table contains the new start and end values after alignment.
\end{Details}
%
\begin{Value}
Extended selection table similar to input data in which time parameters (columns 'start' and 'end') have been tailored to more closely match the start and end of the reference signal.
\end{Value}
%
\begin{Author}\relax
Marcelo Araya-Salas (\email{marceloa27@gmail.com})
\end{Author}
%
\begin{References}\relax

Araya-Salas, M. (2020). baRulho: baRulho: quantifying habitat-induced degradation of (animal) acoustic signals in R. R package version 1.0.0

Clark, C.W., Marler, P. \& Beeman K. (1987). Quantitative analysis of animal vocal phonology: an application to Swamp Sparrow song. Ethology. 76:101-115. 

\end{References}
%
\begin{SeeAlso}\relax
\code{\LinkA{blur\_ratio}{blur.Rul.ratio}}, \code{\LinkA{xcorr}{xcorr}}
\end{SeeAlso}
%
\begin{Examples}
\begin{ExampleCode}
{
# load example data
data("playback_est_unaligned")

# method 1
spcc_align(X = playback_est_unaligned)
}

\end{ExampleCode}
\end{Examples}
\inputencoding{utf8}
\HeaderA{spectral\_blur\_ratio}{Measure blur ratio in the frequency domain}{spectral.Rul.blur.Rul.ratio}
%
\begin{Description}\relax
\code{spectral\_blur\_ratio} measures blur ratio of frequency spectra from signals referenced in an extended selection table.
\end{Description}
%
\begin{Usage}
\begin{verbatim}
spectral_blur_ratio(X, parallel = 1, pb = TRUE, method = 1, ssmooth = 50, 
output = "est", img = FALSE, res = 150, hop.size = 11.6, wl = NULL, 
ovlp = 70, pal = reverse.gray.colors.2, collevels = seq(-60, 0, 5), dest.path = NULL)
\end{verbatim}
\end{Usage}
%
\begin{Arguments}
\begin{ldescription}
\item[\code{X}] object of class 'extended\_selection\_table' created by the function \code{\LinkA{selection\_table}{selection.Rul.table}} from the warbleR package.

\item[\code{parallel}] Numeric vector of length 1. Controls whether parallel computing is applied by specifying the number of cores to be used. Default is 1 (i.e. no parallel computing).

\item[\code{pb}] Logical argument to control if progress bar is shown. Default is \code{TRUE}.

\item[\code{method}] Numeric vector of length 1 to indicate the 'experimental design' for measuring spectrum correlation. Two methods are available:
\begin{itemize}

\item{} \code{1}: compare all signals with their counterpart that was recorded at the closest distance to source (e.g. compare a signal recorded at 5m, 10m and 15m with its counterpart recorded at 1m). This is the default method. 
\item{} \code{2}: compare all signals with their counterpart recorded at the distance immediately before (e.g. a signal recorded at 10m compared with the same signal recorded at 5m, then signal recorded at 15m compared with same signal recorded at 10m and so on).

\end{itemize}


\item[\code{ssmooth}] Numeric vector of length 1 determining the length of the sliding window used for a sum smooth for power spectrum calculation (in kHz). Default is 100.

\item[\code{output}] Character vector of length 1 to determine if an extended selection table ('est') or a list ('list') containing 1) extended selection table and 2) amplitude values is returned.

\item[\code{img}] Logical argument to control if image files in 'jpeg' format containing the images being compared and the corresponding spectra are produced. Default is no images ( \code{FALSE}).

\item[\code{res}] Numeric argument of length 1. Controls image resolution. Default is 150 (faster) although 300 - 400 is recommended for publication/presentation quality.

\item[\code{hop.size}] A numeric vector of length 1 specifying the time window duration (in ms). Default is 11.6 ms, which is equivalent to 512 wl for a 44.1 kHz sampling rate. Ignored if 'wl' is supplied.

\item[\code{wl}] A numeric vector of length 1 specifying the window length of the spectrogram, default 
is NULL. If supplied, 'hop.size' is ignored. Applied to both spectra and spectrograms on image files.

\item[\code{ovlp}] Numeric vector of length 1 specifying the percent overlap between two 
consecutive windows, as in \code{\LinkA{spectro}{spectro}}. Default is 70. Applied to both spectra and spectrograms on image files.

\item[\code{pal}] A color palette function to be used to assign colors in the 
plot, as in \code{\LinkA{spectro}{spectro}}. Default is reverse.gray.colors.2.

\item[\code{collevels}] Numeric vector indicating a set of levels which are used to partition the amplitude range of the spectrogram (in dB) as in \code{\LinkA{spectro}{spectro}}. Default is \code{seq(-60, 0, 5)}.

\item[\code{dest.path}] Character string containing the directory path where the image files will be saved. If NULL (default) then the folder containing the sound files will be used instead.
\end{ldescription}
\end{Arguments}
%
\begin{Details}\relax
Spectral blur ratio measures the degradation of sound as a function of the change in signal energy in the frequency domain, analogous to the blur ratio proposed by Dabelsteen et al (1993) for the time domain (and implemented in \code{\LinkA{blur\_ratio}{blur.Rul.ratio}}). Low values indicate low degradation of signals. The function measures the blur ratio of spectra from signals in which a reference playback has been re-recorded at different distances. Spectral blur ratio is measured as the mismatch between power spectra (expressed as probability density functions) of the reference signal and the re-recorded signal. The function compares each signal type to the corresponding reference signal. The 'signal.type' column must be used to tell the function to only compare signals belonging to the same category (e.g. song-types). Two methods for setting the experimental design are provided. All wave objects in the extended selection table must have the same sampling rate so the length of spectra is comparable.
\end{Details}
%
\begin{Value}
Data frame similar to input data, but also includes a new column (blur.ratio.spectrum)
with the blur ratio values. If \code{img = TRUE} it also returns 1 image file (in 'jpeg' format) for each comparison showing spectrograms of both signals and the overlaid power spectrum (as probability mass functions (PMF)).  Spectrograms are shown within the frequency range of the reference signal and also show vertical lines with the start and end of signals to allow users to visually check alignment. If \code{output = 'list'} the output would a list including the data frame just described and a data frame with spectra (amplitude values) for all signals.
\end{Value}
%
\begin{Author}\relax
Marcelo Araya-Salas (\email{marceloa27@gmail.com})
\end{Author}
%
\begin{References}\relax

Dabelsteen, T., Larsen, O. N., \& Pedersen, S. B. (1993). Habitat-induced degradation of sound signals: Quantifying the effects of communication sounds and bird location on blur ratio, excess attenuation, and signal-to-noise ratio in blackbird song. The Journal of the Acoustical Society of America, 93(4), 2206.

Araya-Salas, M. (2020). baRulho: baRulho: quantifying habitat-induced degradation of (animal) acoustic signals in R. R package version 1.0.0

\end{References}
%
\begin{SeeAlso}\relax
\code{\LinkA{blur\_ratio}{blur.Rul.ratio}}
\end{SeeAlso}
%
\begin{Examples}
\begin{ExampleCode}
{
# load example data
data("playback_est")

# remove ambient selections
playback_est <- playback_est[playback_est$signal.type != "ambient", ]

# using method 1
spectral_blur_ratio(X = playback_est)

# using method 2
spectral_blur_ratio(X = playback_est, method = 2)
}

\end{ExampleCode}
\end{Examples}
\inputencoding{utf8}
\HeaderA{spectral\_correlation}{Measure frequency spectrum correlation}{spectral.Rul.correlation}
%
\begin{Description}\relax
\code{spectral\_correlation} measures frequency spectrum correlation of signals referenced in an extended selection table.
\end{Description}
%
\begin{Usage}
\begin{verbatim}
spectral_correlation(X, parallel = 1, pb = TRUE, method = 1, 
cor.method = "pearson", hop.size = 11.6, wl = NULL, ovlp = 70)
\end{verbatim}
\end{Usage}
%
\begin{Arguments}
\begin{ldescription}
\item[\code{X}] object of class 'extended\_selection\_table' created by the function \code{\LinkA{selection\_table}{selection.Rul.table}} from the warbleR package.

\item[\code{parallel}] Numeric vector of length 1. Controls whether parallel computing is applied by specifying the number of cores to be used. Default is 1 (i.e. no parallel computing).
If \code{NULL} (default) then the current working directory is used.

\item[\code{pb}] Logical argument to control if progress bar is shown. Default is \code{TRUE}.

\item[\code{method}] Numeric vector of length 1 to indicate the 'experimental design' to measure frequency spectrum correlation. Two methods are available:
\begin{itemize}

\item{} \code{1}: compare all signals with their counterpart that was recorded at the closest distance to source (e.g. compare a signal recorded at 5m, 10m and 15m with its counterpart recorded at 1m). This is the default method. 
\item{} \code{2}: compare all signals with their counterpart recorded at the distance immediately before (e.g. a signal recorded at 10m compared with the same signal recorded at 5m, then signal recorded at 15m compared with same signal recorded at 10m and so on).

\end{itemize}


\item[\code{cor.method}] Character string indicating the correlation coefficient to be applied ("pearson", "spearman", or "kendall", see \code{\LinkA{cor}{cor}}).

\item[\code{hop.size}] A numeric vector of length 1 specifying the time window duration (in ms). Default is 11.6 ms, which is equivalent to 512 wl for a 44.1 kHz sampling rate. Ignored if 'wl' is supplied.

\item[\code{wl}] A numeric vector of length 1 specifying the window length of the spectrogram, default 
is NULL. If supplied, 'hop.size' is ignored.

\item[\code{ovlp}] Numeric vector of length 1 specifying the percent overlap between two 
consecutive windows, as in \code{\LinkA{spectro}{spectro}}. Default is 70.
\end{ldescription}
\end{Arguments}
%
\begin{Details}\relax
spectrum correlation measures the similarity of two signals in the frequency domain. The function measures the spectrum correlation coefficients of signals in which a reference playback has been re-recorded at increasing distances. Values range from 1 (identical frequency spectrum, i.e. no degradation) to 0. The 'signal.type' column must be used to indicate the function to only compare signals belonging to the same category (e.g. song-types). The function will then compare each signal type to the corresponding reference signal. Two methods for calculating spectrum correlation are provided (see 'method' argument). Use \code{\LinkA{spectral\_blur\_ratio}{spectral.Rul.blur.Rul.ratio}} to get spectra for plotting.
\end{Details}
%
\begin{Value}
Extended selection table similar to input data, but also includes a new column ('spectrum.correlation')
with the calculated frequency spectrum correlation coefficients.
\end{Value}
%
\begin{Author}\relax
Marcelo Araya-Salas (\email{marceloa27@gmail.com})
\end{Author}
%
\begin{References}\relax

Araya-Salas, M. (2020). baRulho: baRulho: quantifying habitat-induced degradation of (animal) acoustic signals in R. R package version 1.0.0

Apol, C.A., Sturdy, C.B. \& Proppe, D.S. (2017). Seasonal variability in habitat structure may have shaped acoustic signals and repertoires in the black-capped and boreal chickadees. Evol Ecol. 32:57-74.

\end{References}
%
\begin{SeeAlso}\relax
\code{\LinkA{envelope\_correlation}{envelope.Rul.correlation}}, \code{\LinkA{spectral\_blur\_ratio}{spectral.Rul.blur.Rul.ratio}}
\end{SeeAlso}
%
\begin{Examples}
\begin{ExampleCode}
{
# load example data
data("playback_est")

# remove ambient selections
pe <- playback_est[playback_est$signal.type != "ambient", ]

# method 1
spectral_correlation(X = pe)

# method 2
spectral_correlation(X = pe, method = 2)
}

\end{ExampleCode}
\end{Examples}
\printindex{}
\end{document}
